\section{Method}


\subsection{Datastructures}
\begin{itemize}
    \item $K^2$-Raster
    \item R*-Tree
\end{itemize}

\subsection{System Overview}
\todo[inline]{General explanation of the system as a unit}

\subsection{Join Algorithm}
\begin{itemize}
    \item High level overview
    \item Filter functions
    \item checkQuadrant
    \item checkMBR
    \item CombineLists
    \item extractCells
    \item possibly more...
\end{itemize}
\subsubsection*{Filter functions}
One of the main differences between Raven and Beast is that Raven has greater support for joins that filter the result based on pixel-values. We will define a filter as a function that accepts certain pixel-values only. When a filter is applied to a join it means that the result of the join will be only the pixels that overlap with the vector data and whose value is accepted by the filter.

In Raven, the filter is pushed down as far as possible, making it able to disregard pixels that will be filtered away earlier. In Beast, no such functionality exists, this means that the only way to apply a filter on the join is to first compute the full join, then filter the result. 

The simplest useful filter is one that can select values within a certain range. That is, the filter should have 2 values $lo$ and $hi$, and for any value $x$ should match $x$ if and only if $lo \leq x \leq hi$.

Another potentially useful filter is one that matches pixels based on multiple samples from the raster-data. This could be used to match only those pixels that have a red-value in a certain range, a green-value in another range, and a blue-value in a third range.

The pushing down of the filter is accomplished by querying the $K^2$-raster tree for the minimum and maximum pixel value in the area covered by the MBR of a given geometry or collection of geometries. If no values in this section of the $K^2$-raster tree are values accepted by the filter, the geometry in question can be thrown away without having to be processed further. 

Raven supports filter functions that can compute whether a given range of values contain (1) any values that should be accepted by the filter, and (2) any values that should not be accepted by the filter. Any filter-function should therefore implement 2 methods, one for each of these checks. The reason it is done in this way is that the $K^2$-raster tree will give a range of values in the form of a min/max pair for any area queried. This type of filter can therefore be used to figure out if the geometry can be thrown away. This happens if it does not contain any pixels the filter accepts. It can also be used to determine if we can add all pixels within it to the results without having to filter each pixel. This happens when the range does not contain any values the filter will not accept.

The range filter can be implemented in Raven as a filter that for a queried interval $[x_1,x_2]$ returns $x_1 \leq hi \wedge lo \leq x_2$ for check (1) and $x_1 < lo \vee x_2 > hi$ for (2).

In ...\todo{Reference}, which is the paper our system is based on, only filters of this kind are supported. Our more abstract way of implementing these filters allows us to have more control over which values should be accepted. For example, without changing anything other than which filter object is used, we could easily accept values that fall within one of 2 or more given ranges, by simply composing multiple of these filters with 'or' operations. If many ranges are needed to match all the values we want, an interval tree can be used to give a query time of $O(log(n))$, where $n$ is the number of ranges. Alternatively, if the number of bits used by the pixels is not too high, a prefix-sum-array can easily be used to filter values in constant time.

Applying a filter for each sample is also possible using a filter function, but the time taken to join often ends up being the same as using the simpler range filter and the post-processing the result.
\todo[inline]{maybe we can modify the K2-raster slightly to allow filters of the last type to be faster than currently possible}


\subsection{OOCE}
\begin{itemize}
    \item Tiling approach
    \item Streaming of matrices
    \item Only looks at vector shapes that might overlap with the tile
    \item translating vector coordinates to tiles
\end{itemize}
\subsubsection{Distribution}
\todo[inline]{Remove section?}
\begin{itemize}
    \item Spark
    \item Heuristic for splitting the work evenly
\end{itemize}