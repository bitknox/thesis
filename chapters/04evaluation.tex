\section{Evaluation}

\subsection{Workload}
The workloads can be split into two major categories. The real-world datasets and synthetic datasets. The real-world datasets are meant to show real-world performance, allowing us to compare the different approaches. The synthetic sets are meant to test certain aspects of the approach where we control all the parameters, to examine exactly the subsystem or area of interest.
\subsubsection{Real world data}
The following section contains an overview of real world data we are going to use for the evaluation. There are some relevant properties included.
\paragraph{Vector sets} 
\mbox{} % :(
\begin{table}[H]
\begin{tabular}{|l|l|l|}
\hline
Dataset              & Size   & Vertices  \\ \hline
Boundaries           & 8.39MB & 548471    \\ \hline
Protected areas      & 534MB  & 34861822  \\ \hline
Small Woody Features & 1.59GB & 105349748 \\ \hline
\end{tabular}
\caption{asd}
\end{table}
\paragraph{Raster sets}
\mbox{} % :(
\begin{table}[H]
\begin{tabular}{|l|l|l|l|l|}
\hline
Dataset              & Size   & Compression & Bit depth & Colors \\ \hline
GLC                  & 826MB  & None        & 8         & 23     \\ \hline
Treecover            & 534MB  & None        & 8         & 98     \\ \hline
Small Woody Features & 6.04GB & LZW         & 8         & 3      \\ \hline
\end{tabular}
\caption{}
\end{table}
\subsubsection{Synthetic data}
A few previews of the synthetic data for the evaluation can be seen in this section.
\paragraph{Raster (filering workload)}
Selectivity 100\%, 50\%, 25\%, 12.5\%, 6.25\%, 3.125\%, 1.562\%

\begin{figure}[H]
     \centering
     \begin{subfigure}[b]{0.45\textwidth}
         \centering
         \includegraphics[width=\textwidth]{images/04evaluation/selectivity_30.png}
         \caption{30 \%}
         \label{fig:y equals x}
     \end{subfigure}
     \hfill
     \begin{subfigure}[b]{0.45\textwidth}
         \centering
         \includegraphics[width=\textwidth]{images/04evaluation/selectivity_75.png}
         \caption{75\%}
         \label{fig:three sin x}
     \end{subfigure}
     \hfill
        \caption{Comparison of different selectivity for Perlin noise.}
        \label{fig:three graphs}
\end{figure}

\paragraph{Raster (K2 Compression)}
\begin{itemize}
    \item For Voronoi noise, we vary the number of tiles. For each of the tiles, we arrange them, such that each neighboring tile has a different colour. We vary the number of tiles between $2^2, 2^3, ... ,2^8$. An example of two of these tiles can be found in figure \ref{fig:voronoi_example}, where figure \ref{fig:voronoi_16} has 16 tiles and \ref{fig:voronoi_64} has 64 tiles.
    \item Perlin noise. To make our synthetic data resemble real-world data, we utilize Perlin noise to generate the \textit{base} landscape. We then map this landscape to a desired amount of different colors to be used in both bit-depth testing and colour testing. An example of the type of landscapes we generate can be found in figure \ref{fig:perlin_example}.
\end{itemize}

\begin{figure}[H]
     \centering
     \begin{subfigure}[b]{0.45\textwidth}
         \centering
         \includegraphics[width=\textwidth]{images/04evaluation/voronoi16.png}
         \caption{16 tiles}
         \label{fig:voronoi_16}
     \end{subfigure}
     \hfill
     \begin{subfigure}[b]{0.45\textwidth}
         \centering
         \includegraphics[width=\textwidth]{images/04evaluation/voronoi64.png}
         \caption{64 tiles}
         \label{fig:voronoi_64}
     \end{subfigure}
     \hfill
        \caption{Comparison of different amounts of tiles for Voronoi noise.}
        \label{fig:voronoi_example}
\end{figure}


\begin{figure}[H]
    \centering
    \includegraphics[width=0.6\textwidth]{images/04evaluation/perlin.png}
    \caption{Caption}
    \label{fig:perlin_example}
\end{figure}

\paragraph{Raster (bit-depth)}
\begin{itemize}
    \item 8-bit
    \item 16-bit
    \item 24-bit
\end{itemize}


\paragraph{Raster (number of colours)}
$3^2, 3^3, ..., 3^{10}$ colours


\paragraph{Vector (Shape density - Uniform distribution)}
Generating random polygons as distributing them uniformly inside some bounds. Vary number of polygons / area for density.

Number of points static, Irregularity static, spikiness static, avg\_radius static
$10^3, 10^4, ..., 10^{7}$ polygons


\paragraph{Vector (Complex shapes)}
Static number of points, 
\begin{itemize}
    \item Irregularity 0-1
    \item Number of points
\end{itemize}



\subsection{Hardware}
Server specs....

\subsection{Finding optimal parameters}
\begin{itemize}
    \item Find optimal K (might change depending on dataset)
    \item Find optimal number of R*-tree children
    \item Find optimal tilesize 
\end{itemize}

\subsection{Configurations}
\begin{itemize}
    \item All real world vector with all real world raster
    \item Selectivity with square vector covering entire area
    \item Shape density with set Raster set OR synthetic set (static)
    \item Same as above, but with more complex shapes.
    \item Vector set covering entire area, raster varies between: Voronoi nois, Perlin noise \& Large squares. (We might want to tweak params for the noise...)
\end{itemize}
\subsection{Results}
\subsubsection{Real world data}
\subsubsection{Synthetic data}
\subsubsection{Comparison with other systems}
\subsection{Implementation details}

factors we gain performance
\begin{itemize}
    \item our cached compressed index.
    \item we can return a whole pixel range as result because we know it covers a single value.
    \item subsequent joins of compressed raster data.
    \item No sorting step for our pixel ranges.
\end{itemize}

\subsubsection{Benchmarking framework}
\todo[inline]{Describe how benchmarking framework works}

\subsubsection{Runners}
\todo[inline]{Mention why beast is not here }
\paragraph{Raptor runner}
\paragraph{Raven runner}