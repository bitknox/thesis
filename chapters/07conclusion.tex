\section{Conclusion \& Future Work}
\label{sec:conclusion-future-work}
In this section we will conclude the report and give directions to what future work might be. 

\subsection{Conclusion}
\label{sec:conclusion}
\begin{itemize}
    \item reflection on results
    \item average differences between us and beast
    \item filtering is fast
    \item result types
    \item low memory consumption
    \item comparison with existing systems
\end{itemize}
\todo[inline]{Finish conclusion}
\subsection{Future Work}
\label{sec:future_work}
In \citet{k2raster} there is a hybrid version of the \kraster that improves compression and the get window function of the data structure. The speed improvement of the data structure is most notable for a dataset with many different values and this would in theory improve how quickly we could retrieve pixel values for a join with no filter. The cost using this variant of \kraster is only the build time for datasets with many values. The get window function from their evaluation is most comparable to what happens in a join with no filter. The hybrid version is about three times faster for each window. Adjusting for the time it takes to create results (figure \ref{fig:result_creation_time}, using the hybrid version might be able to beat \raptorjoin for no filter joins of datasets similar to Treecover.

We think that it is possible to compress the result even more because sometimes the pixel ranges start and stop at the position in the horizontal axis. If it is possible to create one large rectangle instead of individual lines, we could save a great deal of work. This is mostly going matter for raster data with large areas of the same values. When the system is in the routine of finding the pixel ranges, it will know what \kraster node it has to join within. And with that, it follows what the maximum and minimum values lie within this node. Therefore we can skip further traversal of the \kraster if all the values of the node is the same.

Using the metadata of all the raster data, we should be able to partition the vector data without building a \str. This would also allow the vector data to be read in parallel because each polygon only has to be added to the correct partitions. This would require synchronization because the collections for each partition would need to be shared between threads.