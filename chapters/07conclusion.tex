\section{Conclusion \& Future Work}
\label{sec:conclusion-future-work}
In this section we will conclude the report and give directions to what the future work might be. 

\subsection{Conclusion}
In this thesis, we introduced the system \raven, a system for efficiently computing raster-vector joins tailored for single-machine use. \raven takes a different approach to the state-of-the-art system \raptorjoin by allowing efficient joins where the results are filtered based on raster values.

We compared our \ravenjoin algorithm to \raptorjoin, and saw that if \ravenjoin is allowed to use the cache it builds, it outperforms \raptorjoin on most datasets. Even if we consider a situation where the cache has not already been built, \ravenjoin can still outperform \raptorjoin if the same raster data is joined with different vector data at least 2-3 times.

In our comparison with \raptorjoin, we used 3 different raster datasets. For \textit{GLC2000}, we are 19\% faster on average for filtered joins, and 9\% slower for unfiltered joins. The reason we are slower for unfiltered joins is that we spend time building an index of the vector data, something \raptorjoin does not do.
For \textit{Treecover}, we perform filtered joins 88\% faster, and around 1\% faster for unfiltered joins. For the unfiltered joins we are hurt by the very scattered pixel values of \textit{Treecover}.
For \textit{Small Woody Features (raster)}, we are 92\% faster for filtered joins and 86\% faster for unfiltered ones. The majority of the difference comes from the fact that \textit{Small Woody Featues} needs to be decompressed when read, which hurts \raptorjoin but not \ravenjoin.
Overall, we saw that for filtered joins, \ravenjoin performs way better than \raptorjoin. This is because the filter is integrated into the join algorithm, allowing \ravenjoin to disregard parts of the raster data that should not be joined because of the filter much earlier than \raptorjoin can.

We presented 3 types of results \ravenjoin can produce; Individual pixels with values, Ranges with no values, and one where consecutive pixels with the same value are compressed into a range with that value. We saw that our compressed value ranges, which store the same information as individual pixels, performs significantly better. They are both faster and more memory friendly. We also saw that the no value range type is significantly faster still, with the caveat that they only contain coordinates.

\ravenjoin especially shines when it comes to data that is compressed and therefore slow to read, since the caches it generates are much faster to read and take up little extra space. Datasets with few or very grouped colours are also advantageous for us, as our compressed result format helps to reduce both memory consumption and running time. This, along with our ability to use any partition size allows us to have a significantly better memory profile than \raptorjoin. 

We saw that the time taken by \ravenjoin mostly depends on the amount of results it must create. We were also able to both prove and show experimentally that our new approach to searching for values in the \kraster tree, is more efficient than those that existed previously.

The size of the cache behaves as expected, having a dependency on the size of the values present in the image, as well as how grouped the values are. 

\subsection{Future Work}
\label{sec:future_work}
In \citet{k2raster} there is a hybrid version of the \kraster that improves compression and the get window function of the data structure. The speed improvement of the data structure is most notable for a dataset with many different values and this would in theory improve how quickly we could retrieve pixel values for a join with no filter. The cost using this variant of \kraster is only the build time for datasets with many values. The get window function from their evaluation is most comparable to what happens in a join with no filter. The hybrid version is about three times faster for each window. Adjusting for the time it takes to create results (figure \ref{fig:result_creation_time}, using the hybrid version might be able to beat \raptorjoin for no filter joins of datasets similar to \textit{Treecover}.

We think that it is possible to compress the result even more because sometimes the pixel ranges start and stop at the same position on the horizontal axis. If it is possible to create one large rectangle instead of individual lines, we could save a great deal of work. This is mostly going matter for raster data with large areas of the same values. When the system is in the routine of finding the pixel ranges, it will know what \kraster node it has to join within. And with that, it follows what the maximum and minimum values lie within this node. Therefore we can skip further traversal of the \kraster if all the values of the node is the same.