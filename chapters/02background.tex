\section{Background} \label{sec:background}
This section describes the necessary background information needed to understand the data types and data structures that are handled by \raven.

\subsection{Raster Data}
Raster data is one of the two types of data our algorithm uses as input.
It is data laid out in grids of given dimensions, where each cell or pixel holds a value \cite{GIS-Essentials}. Typically this is a 2D grid of values and in the context of geographical data, this represents an area of the earth. The raster data can be referenced as an image of rows and columns. The size of the image can be found by multiplying the number of rows with the number of columns. See figure \ref{fig:raster-model} for an example of the raster model. For simplicity, we will assume that raster data is 2-dimensional and has a width and a height.
To map the raster data to the face of the earth, metadata is added with transformation and projection information. In combination, it is possible to tell what area the whole raster data covers. The discrete cells have certain lengths, usually given in meters. This is also referred to as the resolution. Images with fine resolutions take up much more space compared to images with worse resolution covering the same area.

\begin{figure}[H]
    \centering
    \includegraphics[height=6cm]{images/02background/raster_dataset.png}
    \caption{Illustration of the raster model. Source: \url{https://docs.qgis.org/3.34/en/docs/gentle_gis_introduction/raster_data.html\#id1}}
    \label{fig:raster-model}
\end{figure}


\subsection{Vector Data}
The other of the two types of data used as input is vector data.
The vector data contains points of X and Y coordinate pairs \cite{GIS-Essentials}. Each point represent a location on the earth and therefore a vector dataset can easily cover the whole earth with a few points. This makes the typical file size of vector model much smaller, compared to the raster model. An important difference between raster and vector data is that raster data is very good at describing noisy or continuous data. On the other hand, vector data cannot do this well, since it would need a separate shape for each group in the data.

Connecting the points will form lines, which again can be connected to form polygons that are enclosed shapes or geometries (See figure \ref{fig:vector-model}. On the basis of polygons always being closed, we can say that a horizontal line, crossing through the polygon, must intersect an even number of times. This is useful for rasterisation of the vector data.

\begin{figure}[H]
    \centering
    \includegraphics[height=6cm]{images/02background/polygon_feature.png}
    \caption{Illustration of the vector model. Source: \url{https://docs.qgis.org/3.34/en/docs/gentle_gis_introduction/vector_data.html\#id4}}
    \label{fig:vector-model}
\end{figure}

\subsection{Spatial Joins}

A spatial join can be defined to be a join based on the location of data points of spatial data. More specifically the data is joined by the distance or physical intersection. For raster and vector data, a given polygon is joined with the pixels, that are inside the polygon. Whether a pixel overlaps with a given polygon can be up to interpretation. For example, an overlap could mean that a pixel is completely within the polygon or that just the center point of the pixel is contained.

To join raster and vector data, it is important that the projections of the datasets are the same. The projection information attached to the data, is known as Coordinate Reference System (\textit{CRS}). The datasets are often projected differently and it would be inefficient to convert raster to vector or vice versa. The converted dataset would be inflated in size and would require reprojection for each new dataset if the CRS does not match. Raster data is not easily reprojected because each pixel is converted to a point which takes up more space. This reprojection is not feasible for large datasets \cite{singla2020raptor}. Instead we want to convert the vector data to the coordinate system of the raster data. This is much cheaper since vector does not have as many points as raster and each point already has a designated location. This comes with the assumption that the lines of the vector data are not significantly different between the two projections.

%new
The output of the join is all the pixels that are within the geometries. Each geometry will have a list of belonging joined pixels, and if two geometries cover the same area, that area of pixels will be contained in both their lists.
%new

\subsection{State-of-the-art Systems \& Algorithms}

\subsubsection{\beast}
\beast is a state-of-the-art system, that can join raster and vector data on a distributed level and it utilizes the \raptorjoin algorithm to do this. We are going to compare our method with this implementation of \raptorjoin.

\raptorjoin is designed to do distributed raster and vector joins. The algorithm splits images into smaller sections (tiles), that can be distributed across spark nodes. The tiles \raptorjoin uses are those defined within the raster image. For each tile, intersections between line segments of the polygons and raster data rows are found, to create pixel ranges. These ranges can then be used as scan lines, to read the exact pixels that are needed for the join. The join only uses the metadata to create the pixel ranges, which means the raster data is not read before it is strictly necessary. Pixel ranges consists of (r, x1, x2), where r is the row of the pixel range, and x1 and x2 are the start and end points. Once pixel ranges have been found, they are sorted, such that reading values from the raster files, are accessed in the in same order, as in memory.

The final step of \raptorjoin is to read the data and create results. The results consists of tuples (gid, rid, x, y, m) where gid is the id of the geometry. rid is the id of the raster tile. (x, y) is the position within the raster image. m is the value of the pixel.

