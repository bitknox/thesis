\section{Background} \label{sec:background}

\subsection{Raster Data}

Raster data is data laid out in grids of given dimensions where each cell or pixel holds a value \cite{GIS-Essentials}. Typically this is a 2D grid of values and in the context of geographical data, this represents an area of the earth. The raster data can be referenced as an image of rows and columns and size of the image is how many rows and columns there are. See figure \ref{fig:raster-model} for an example. For simplicity, the size is a width and a height.
To map the raster data to the face of the earth, metadata is added with transformation and projection information. In combination it is possible to tell what area the whole raster data covers. Given the width and height and the physical area, the discrete cells have certain lengths of meters. This is also referred to as the resolution. A relatively small resolution would be in kilometers and high resolutions are in meters. It might become obvious that if the resolution is one meter, the size of the data grows quite rapidly in proportion with the area the set covers.

\begin{figure}[H]
    \centering
    \includegraphics[height=6cm]{images/02background/raster_dataset.png}
    \caption{Illustration of the raster model. Source: \url{https://docs.qgis.org/3.34/en/docs/gentle_gis_introduction/raster_data.html\#id1}}
    \label{fig:raster-model}
\end{figure}


\subsection{Vector Data}

At a basic level, the vector data model is defined as points of X and Y coordinate pairs \cite{GIS-Essentials}. Each point represent a location on the earth and whole vector dataset can easily cover the whole earth with a few points. This makes the vector model much smaller compared to the raster model. On the other hand, vector data cannot be used for much else than to define an area. Connecting the points will form lines which can be connected to form polygons that are enclosed shapes (See figure \ref{fig:vector-model}. On the basis of polygons always being closes we can say that a horizontal line crossing through the polygon, it must intersect an even number of times. 

\begin{figure}[H]
    \centering
    \includegraphics[height=6cm]{images/02background/polygon_feature.png}
    \caption{Illustration of the vector model. Source: \url{https://docs.qgis.org/3.34/en/docs/gentle_gis_introduction/vector_data.html\#id4}}
    \label{fig:vector-model}
\end{figure}

\subsection{Spatial Joins}

A spatial join can be defined to be a join based on the location of data points of spatial data. More specifically the data is joined by the distance or physical intersections. For raster and vector data, a given polygon is joined with the pixels that are overlapping with the polygon. There are of course some definitions to be aware of, regarding when to say something overlaps. For example, if a pixel is completely within the polygon or if it is just the center of the pixel.

To join raster and vector data, it is important that the projections of the datasets to get the same coordinate system. The projection information attached to the data is known as Coordinate Reference System (CRS) and datasets can be using many different systems. This is why it would be very inefficient to convert the whole dataset to either vector or raster data. The converted dataset would be inflated in size and would require reprojection for each new dataset. Since it is cumbersome to reproject raster data, we instead want to convert the vector data to the coordinate system of the raster data. This is much cheaper since vector does not have as many points as raster. This comes with the assumption that the edges of the vector data are not very different between the two projections.

%new
The output of the join is all the pixels that are within the geometries. Each geometry will have a list of belonging joined pixels and if two geometries cover the same area, the lists will simply be the same.
%new

\subsection{State-of-the-art Systems \& Algorithms}

\subsubsection{\beast}
\beast is a state-of-the-art system that can join raster and vector data on a distributed level. It utilizes the \raptorjoin algorithm to do this. \beast contains the implementation of \raptorjoin that we are going to compare our method to.

\raptorjoin is designed to do distributed raster and vector joins by pixel ranges that are contained within the vector shapes. The algorithm splits images into smaller sections(tiles), that can be distributed across spark nodes. For each tile, intersections between line segments of the polygons and raster data rows are found, to create pixel ranges. These ranges can then be used as scan lines to read the exact pixels that are needed for the join. The join only uses the metadata to create the pixel ranges which means the raster data is not read before it is strictly necessary. Pixel ranges consists of (r, x1, x2),
where r is the row of the pixel range and x1 and x2 are the start and endpoints. Once pixel ranges have been found, they are sorted, such that when reading values from the raster files, they are accessed in the same order they have in memory.

The final step of \raptorjoin is to read the data and create results. The results consists of tuples (gid, rid, x, y, m) where gid is the id of the geometry. Rid the id of the raster tile. (x,y) The position within the raster image with id rid and m the which holds the value of the pixel.

