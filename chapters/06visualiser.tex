\section{Visualizer} \label{sec:visualiser}

The Raven system contains a visualizer which is helpful for checking correctness and finding bugs in the implementation. If a join takes longer or shorter than expected, the visualizer can also help make it clearer why that happened. Any drawing done by the visualizer can use at least either a single colour or randomly generated colours. The visualizer supports the following operations
\begin{itemize}
    \item Drawing the result created by the \ravenjoin algorithm. In addition to the two other colour types, this can be also done using the original colour scheme from the raster file.
    \item Drawing the nodes of the \kraster tree.
    \item Drawing the MBR of the nodes of the \rstar.
    \item It can also draw the vector data on top of any of the previous.
\end{itemize}

In figure \ref{fig:join-example} it is showcased how a join is visualized. Every cell of the raster data inside the polygons are reconstructed in an image. 


\begin{figure}[H]
    \centering   %
\setkeys{Gin}{width=\linewidth}
\begin{subfigure}{.45\textwidth}
  \includegraphics{images/06visualiser/GLC raw.png}
  \caption{An area of the GLC2000 raster dataset corresponding to Europe.}
  \label{fig:BC}
\end{subfigure}
\hfil
\begin{subfigure}{.45\textwidth}
  \includegraphics{images/06visualiser/protected areas.png}
  \caption{The vector set protected areas.\\ \mbox{}}
  \label{fig:DC}
\end{subfigure}

\begin{subfigure}{.8\textwidth}
  \includegraphics{images/06visualiser/GLC join.png}
  \caption{The result of joining that area with the Protected Areas vector set.}
  \label{fig:EC}
\end{subfigure}%
\caption{A visualisation of a join, highlighting the ability to visualise results using the original colours of the image}
\label{fig:join-example}
\end{figure}