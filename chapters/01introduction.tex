\section{Introduction} \label{sec:intro}

\begin{itemize}
    \item this is a continuation of our research project
    \item mention the major improvements over the Raven from our research project 
    \item mention the components of the system (overview)
\end{itemize}

This thesis builds on top of our research project where we started working on the Raven system. In the research project, our system was able to build a \kraster tree, build an \rtree and compute a join for some datasets only. We also had a visualiser that was able to colour the pixels present in the results one colour and those that were not present a different colours. The visualiser was also able to overlay the vector data onto its drawing. The results generated by this version of \ravenjoin were not complete, in the sense that that it was not possible to retrieve pixel-values from the results.

\subsection{System Overview}
The system consists of 4 major components
\paragraph{Raptor} is our implementation of the RaptorJoin algorithm running outside of Spark for better performance on our setup. It is based on the implementation by \href{https://bitbucket.org/bdlabucr/beast/src/master/}{Beast}.
\paragraph{Eagle} is a tool which allows users to easily run benchmarks of spatial joins and automatically plot the results (see \ref{sec:eagle}).
\paragraph{Swift} is a tool for generating artificial vector and raster data (see \ref{sec:swift}).
\paragraph{Raven} is the main component of our system. It consists of the \ravenjoin algorithm (see \ref{sec:raven}, as well as a tool for visualising the results of a join (see \ref{sec:visualiser}).

\todo[inline]{ Create figure :)}