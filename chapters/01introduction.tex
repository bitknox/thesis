\section{Introduction}

\begin{itemize}
    \item this is a continuation of our research project
    \item mention the major improvements over the Raven from our research project 
    \item mention the components of the system (overview)
\end{itemize}

Since the search performance of the r-tree is impacting the join time, we want to provide a metric of how efficient the search is. The search time is mostly affected by how the directorial nodes overlap each other. To compare how these overlaps differ between different r-trees, an overlap ratio is calculated using the definition from  \citet{abdullahalmamun2022airtree}. If the ratio is 1.0, it means the query only traversed leaf nodes that had matches of entries. In table \ref{tab:overlap_ratio} there are overlap ratios for the STR, r*- and r-tree where the queried areas are the bounds of the individual "treecover" images.

\begin{table}[H]
    \centering
    \begin{tabular}{|c|c|c|c|}\hline  
         &  STR&  \rtree& r-tree
\\ \hline  
         Boundaries&  41.01\% &  49.27\% & 36.84\%
\\ \hline  
         Proteccted Areas&  26.68\%&  40.11\% & 25.82\%
\\ \hline  
         Woody&  25.05\%&  27.48\%& 25.24\%
\\ \hline 
    \end{tabular}
    \caption{Overlap ratio for different types of r-trees with 3 different vector data sets. The minimum and maximum sub nodes for the r-tree nodes are 2 and 8 respectively.}
    \label{tab:overlap_ratio}
\end{table}