\section{Introduction} \label{sec:intro}

\begin{itemize}
    \item this is a continuation of our research project
    \item mention the major improvements over the \raven from our research project 
    \item mention the components of the system (overview)
\end{itemize}

In \citetitle{rastervectorstudy} \cite{rastervectorstudy} some of the important use cases of raster and vector joins are highlighted such as preventing wildfires. Zones of interest are created with vector data, to focus the view of tons of satellite images. New datasets that are more comprehensible and manageable are possible with joins of large raster and vector datasets. The problem is how to process this large quantity of data in a feasible way. A few methods have been developed to tackle this problem over the recent years. The common approach is to join one vector dataset with one raster dataset as fast as possible. However, it would require an equal amount of work to join with a new vector dataset with the same raster data because it cannot be kept in memory. It would not be hard to imagine that the zones could undergo iterations that would require additional joins. We want to explore the use of data structures to make consecutive raster and vector joins faster in the context of a system.

%NEW
Another common operation within the realm of spatial joins is filtering \todo[inline]{citation needed}. In current systems such as \beast \cite{Beast} that implement the \raptorjoin \cite{raptorjoin} algorithm, filtering is done after the join is computed. Our approach explores pushing the filter down such that speed increases as selectivity decreases.

%NEW

This thesis builds on top of our research project \cite{ResearchProj} where we started working on the \raven system. In the research project, our system was able to build a \kraster tree, build an \rstar and compute a join for some datasets only. We also had a visualiser that was able to colour the pixels present in the results one colour and those that were not present a different colours. The visualiser was also able to overlay the vector data onto its drawing. The results generated by this version of \ravenjoin were not complete, in the sense that that it was not possible to retrieve pixel-values from the results.

\newpage
\subsection{System Overview}
The system consists of 4 major components
\paragraph{\raven} is the main component of our system. It consists of the \ravenjoin algorithm (see section \ref{sec:raven}, as well as a tool for visualizing the results of a join (see section \ref{sec:visualiser}). This report will cover the \raven component in detail.
\paragraph{Raptor} is our implementation of the RaptorJoin algorithm running outside of Spark for better performance on our setup. It is based on the implementation by \href{https://bitbucket.org/bdlabucr/beast/src/master/}{\beast}. (see section \ref{sec:raptor-join})
\paragraph{Eagle} is a tool which allows users to easily run benchmarks of spatial joins and automatically plot the results (see section \ref{sec:eagle}).
\paragraph{Swift} is a tool for generating artificial vector and raster data (see section \ref{sec:swift}).


\begin{figure}[H]
    \centering
    \includegraphics[width=\textwidth]{images/01introduction/system.png}
    \caption{Overview of components and how they relate to each other}
    \label{fig:system}
\end{figure}